% -----------------------------------------------
% Template for ICMC SMC 2014
% adapted and corrected from the template for SMC 2013,  which was adapted from that of  SMC 2012, which was adapted from that of SMC 2011
% -----------------------------------------------

\documentclass{article}
\usepackage{icmcsmc2014}
\usepackage[]{biblatex}
\addbibresource{SpatDIF.bib}
\usepackage{times}
\usepackage{ifpdf}
\usepackage[english]{babel}
%\usepackage{cite}

%%%%%%%%%%%%%%%%%%%%%%%% Some useful packages %%%%%%%%%%%%%%%%%%%%%%%%%%%%%%%
%%%%%%%%%%%%%%%%%%%%%%%% See related documentation %%%%%%%%%%%%%%%%%%%%%%%%%%
%\usepackage{amsmath} % popular packages from Am. Math. Soc. Please use the 
%\usepackage{amssymb} % related math environments (split, subequation, cases,
%\usepackage{amsfonts}% multline, etc.)
%\usepackage{bm}      % Bold Math package, defines the command \bf{}
%\usepackage{paralist}% extended list environments
%%subfig.sty is the modern replacement for subfigure.sty. However, subfig.sty 
%%requires and automatically loads caption.sty which overrides class handling 
%%of captions. To prevent this problem, preload caption.sty with caption=false 
%\usepackage[caption=false]{caption}
%\usepackage[font=footnotesize]{subfig}


%user defined variables
\def\papertitle{Implementing SpatDIF: Practical Application in Audio Software}
\def\firstauthor{Jan C. Schacher}
\def\secondauthor{Chikashi Miyama}
\def\thirdauthor{Trond Lossius}

% adds the automatic
% Saves a lot of ouptut space in PDF... after conversion with the distiller
% Delete if you cannot get PS fonts working on your system.

% pdf-tex settings: detect automatically if run by latex or pdflatex
\newif\ifpdf
\ifx\pdfoutput\relax
\else
   \ifcase\pdfoutput
      \pdffalse
   \else
      \pdftrue
\fi

% \ifpdf % compiling with pdflatex
  \usepackage[pdftex,
    pdftitle={\papertitle},
    pdfauthor={\firstauthor, \secondauthor, \thirdauthor},
    bookmarksnumbered, % use section numbers with bookmarks
    pdfstartview=XYZ % start with zoom=100% instead of full screen; 
                     % especially useful if working with a big screen :-)
   ]{hyperref}
  %\pdfcompresslevel=9

  \usepackage[pdftex]{graphicx}
  % declare the path(s) where your graphic files are and their extensions so 
  %you won't have to specify these with every instance of \includegraphics
  \graphicspath{./figures/}
  \DeclareGraphicsExtensions{.pdf,.jpeg,.png}

  \usepackage[figure,table]{hypcap}
	
% \else % compiling with latex
%   \usepackage[dvips,
%     bookmarksnumbered, % use section numbers with bookmarks
%     pdfstartview=XYZ % start with zoom=100% instead of full screen
%   ]{hyperref}  % hyperrefs are active in the pdf file after conversion
% 
%   \usepackage[dvips]{epsfig,graphicx}
%   % declare the path(s) where your graphic files are and their extensions so 
%   %you won't have to specify these with every instance of \includegraphics
%   \graphicspath{./figures/}
%   % \DeclareGraphicsExtensions{.eps}
% 
%   \usepackage[figure,table]{hypcap}
% \fi

%setup the hyperref package - make the links black without a surrounding frame
\hypersetup{
    colorlinks,%
    citecolor=black,%
    filecolor=black,%
    linkcolor=black,%
    urlcolor=black
}


% Title.
% ------
\title{\papertitle}

% Authors
% Please note that submissions are NOT anonymous, therefore 
% authors' names have to be VISIBLE in your manuscript. 
%
% Single address
% To use with only one author or several with the same address
% ---------------
%\oneauthor
%   {\firstauthor} {Affiliation1 \\ %
%     {\tt \href{mailto:author1@smcnetwork.org}{author1@smcnetwork.org}}}

%Two addresses
%--------------
% \twoauthors
%   {\firstauthor} {Affiliation1 \\ %
%     {\tt \href{mailto:author1@smcnetwork.org}{author1@smcnetwork.org}}}
%   {\secondauthor} {Affiliation2 \\ %
%     {\tt \href{mailto:author2@smcnetwork.org}{author2@smcnetwork.org}}}

% Three addresses
% --------------
 \threeauthors
   {\firstauthor} {Zurich University of the Arts\\
   Institute for Computer Music\\ and Sound Technology\\
     {\tt \href{jan.schacher@zhdk.ch}{jan.schacher@zhdk.ch}}}
   {\secondauthor} {Affiliation2 \\ %
     {\tt \href{mailto:me@chikashi.net}{author2@smcnetwork.org}}}
   {\thirdauthor} { Affiliation3 \\ %
     {\tt \href{mailto:author3@smcnetwork.org}{author3@smcnetwork.org}}}


% ***************************************** the document starts here ***************
\begin{document}
%
\capstartfalse
\maketitle
\capstarttrue
%
\begin{abstract}
The development and specification of SpatDIF, the spatial sound descriptor interchange format, is complemented with an actual software implementation in order to become usable in various environments. 
In this article, the current state in the development of a software library called `SpatDIFlib' is discussed.
The design principles derived from the concepts and specifications of SpatDIF, the class structure of the library, and code demonstrating its usage is presented.
Furthermore, an application that utilises the library is introduced as an exemplary use case.
\end{abstract}
%
\section{Background}

In this article we present the development of a software tool aimed at simple integration of SpatDIF into existing software.
The concepts and guidelines are implemented in a C-Library and applied in an example surround-playback application.

SpatDIF, the Spatial Sound Description Interchange Format, presents a structured syntax for describing spatial sound information, addressing the different tasks involved in creating and performing spatial sound.

The goal of the SpatDIF approach is to simplify and enhance the methods of creating and exchanging spatial sound content. 
SpatDIF proposes a simple, minimal, and extensible format as well as best-practice implementations for storing and transmitting spatial sound scene descriptions. 
It encourages portability and the exchange of compositions between venues with different surround sound infrastructures. 
SpatDIF also fosters collaboration between artists such as composers, musicians, sound installation artists, and sound designers, as well as researchers in the fields of acoustics, musicology, sound engineering and virtual reality.

SpatDIF is developed as a collaborative effort and has evolved over a number of years. 
The community and all related information can be found at {\it www.spatdif.org}.

SpatDIF was coined in \citeyear{peters_caa07} \cite{peters_caa07} when Peters stated the necessity for a format to describe spatial sound scenes in a structured way, since all of the available spatial rendering systems used self-contained syntax and data-formats at that time. 
Through a panel discussion \cite{2008ICMCpanel, Peters:2008spatdif} and other meetings and workshops, the concept of SpatDIF has been extended, refined, and consolidated.

After a long and thoughtful process, the SpatDIF specification was informally presented to the spatial sound community at the ICMC in Huddersfield in August 2011 and at a workshop at the TU-Berlin in September 2011.
The responses in these meetings suggested the urgent need for a lightweight and easy to implement spatial sound scene standard, which could contrast the complex MPEG-4 scene description specification \cite{scheirer1999audiobifs}.
In addition, several functions necessary to make this lightweight standard become functional, such as the capability of dealing with temporal interpolation of scene descriptors as described, were introduced in \cite{Peters:2013SpatDifCMJ}.


One of the guiding principles for SpatDIF is the idea that authoring and rendering of spatial sound may occur at completely separate times and places, and be executed with tools whose capabilities cannot be known in advance. 
SpatDIF formulates a concise semantic structure that is capable of carrying all the relevant information, without being tied to a specific implementation, thought-model or technical method. 
SpatDIF is a syntax rather than a programming interface or file-format and may be represented in any of the structured mark-up languages or message systems that are in use today or in the future. 
It describes only those aspects required for the storage and transmission of \emph{spatial sound information}.
Because a complete work typically contains aspects that are outside the realm of such spatial sound description, SpatDIF judiciously provides interfaces to link these aspects to the spatial dimensions.
% These are judiciously addressed only to the extent necessary for linking these external sound elements to the descriptions of the spatial dimension. 
%For example, the description of media-streams in SpatDIF is necessary to define the storage location of the audio objects in the scene.
For example, to be able to render the audio objects with the correct audio content in the spatial scene, the storage location of the unrendered audio content needs to be defined via SpatDIF's media resources.
%For example, the description of media-streams in the Media extension is necessary in order to describe from where the sounding content of the scene originates.
\\
After establishing a coherent specification with example use cases in textual form only, the next development step is the implementation of software that embodies the specified concepts.
For this purpose a platform-independent software library was designed and is being implemented. 
Additionally, to demonstrate the usage of SpatDIF, an example application which utilizes this library was created. This application features SpatDIF file-handling and  rendering of SpatDIF sound scenes into multichannel sound files.

By providing a software library rather than just a complete software application, implementations in many different software environments are facilitated, which is one of the strategic goals of the project.


\section{Class Structure}\label{sec:class_structure}
Half page??

\section{Introduction to externals}
Even though the SpatDIF syntax is an implementation-independent specification, rather than a concrete software interface, the actual value of using it only becomes evident in concrete applications. Although SpatDIF was developed with a number of different usage scenarios in mind, the ones most closely associated with these authors' practices are electro-acoustic surround audio compositions for concerts or installations or real-time spatialisation in computer music performance.
Therefore the primary application for the SpatDIF library lies in an implementation in real-time capable audio software, such as MaxMSP or Pure Data.

\subsection{Concepts and Practical usage}\label{subsec:concepts}
In order to explore the methods and actual handling of the libspatdif in a real situation a dual testbed was implemented as externals for MaxMSP and Pure Data. In addition an example application with a limited feature-set was tested in OpenFrameworks, mainly to establish and test a workflow purely in C++.

When implementing thelibspatdif in these software environments a number of decisions had to be made about the level of encapsulation and the degree of control given to the library on the one hand and the client-application on the other hand. The main idea is that libspatdif serves as a memory store for the client (external host) and provides an interface to get or set the events present within the scene. 




\subsection{Playback}\label{sec:Playback}
example text

\subsection{Recording}\label{sub:body}
example text

\section{Future work}
example text


\begin{acknowledgments}
example text
\end{acknowledgments} 

%%%%%%%%%%%%%%%%%%%%%%%%%%%%%%%%%%%%%%%%%%%%%%%%%%%%%%%%%%%%%%%%%%%%%%%%%%%%%
%bibliography here
% \bibliography{SpatDIF.bib}
\printbibliography

\end{document}
