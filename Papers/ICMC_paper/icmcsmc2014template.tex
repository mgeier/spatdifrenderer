% -----------------------------------------------
% Template for ICMC SMC 2014
% adapted and corrected from the template for SMC 2013,  which was adapted from that of  SMC 2012, which was adapted from that of SMC 2011
% -----------------------------------------------

\documentclass{article}
\usepackage{icmcsmc2014}
\usepackage[]{biblatex}
\addbibresource{SpatDIF.bib}
\usepackage{times}
\usepackage{ifpdf}
\usepackage[english]{babel}
\sloppy
%\usepackage{cite}

%%%%%%%%%%%%%%%%%%%%%%%% Some useful packages %%%%%%%%%%%%%%%%%%%%%%%%%%%%%%%
%%%%%%%%%%%%%%%%%%%%%%%% See related documentation %%%%%%%%%%%%%%%%%%%%%%%%%%
%\usepackage{amsmath} % popular packages from Am. Math. Soc. Please use the 
%\usepackage{amssymb} % related math environments (split, subequation, cases,
%\usepackage{amsfonts}% multline, etc.)
%\usepackage{bm}      % Bold Math package, defines the command \bf{}
%\usepackage{paralist}% extended list environments
%%subfig.sty is the modern replacement for subfigure.sty. However, subfig.sty 
%%requires and automatically loads caption.sty which overrides class handling 
%%of captions. To prevent this problem, preload caption.sty with caption=false 
%\usepackage[caption=false]{caption}
%\usepackage[font=footnotesize]{subfig}


%user defined variables
\def\papertitle{Introducing Libspatdif and its Practical Applications in Audio Software}
\def\firstauthor{Jan C. Schacher}
\def\secondauthor{Chikashi Miyama}
\def\thirdauthor{Trond Lossius}

% adds the automatic
% Saves a lot of ouptut space in PDF... after conversion with the distiller
% Delete if you cannot get PS fonts working on your system.

% pdf-tex settings: detect automatically if run by latex or pdflatex
\newif\ifpdf
\ifx\pdfoutput\relax
\else
   \ifcase\pdfoutput
      \pdffalse
   \else
      \pdftrue
\fi

% \ifpdf % compiling with pdflatex
  \usepackage[pdftex,
    pdftitle={\papertitle},
    pdfauthor={\firstauthor, \secondauthor, \thirdauthor},
    bookmarksnumbered, % use section numbers with bookmarks
    pdfstartview=XYZ % start with zoom=100% instead of full screen; 
                     % especially useful if working with a big screen :-)
   ]{hyperref}
  %\pdfcompresslevel=9

  \usepackage[pdftex]{graphicx}
  % declare the path(s) where your graphic files are and their extensions so 
  %you won't have to specify these with every instance of \includegraphics
  \graphicspath{./figures/}
  \DeclareGraphicsExtensions{.pdf,.jpeg,.png}

  \usepackage[figure,table]{hypcap}
	
% \else % compiling with latex
%   \usepackage[dvips,
%     bookmarksnumbered, % use section numbers with bookmarks
%     pdfstartview=XYZ % start with zoom=100% instead of full screen
%   ]{hyperref}  % hyperrefs are active in the pdf file after conversion
% 
%   \usepackage[dvips]{epsfig,graphicx}
%   % declare the path(s) where your graphic files are and their extensions so 
%   %you won't have to specify these with every instance of \includegraphics
%   \graphicspath{./figures/}
%   % \DeclareGraphicsExtensions{.eps}
% 
%   \usepackage[figure,table]{hypcap}
% \fi

%setup the hyperref package - make the links black without a surrounding frame
\hypersetup{
    colorlinks,%
    citecolor=black,%
    filecolor=black,%
    linkcolor=black,%
    urlcolor=black
}


% Title.
% ------
\title{\papertitle}

% Authors
% Please note that submissions are NOT anonymous, therefore 
% authors' names have to be VISIBLE in your manuscript. 
%
% Single address
% To use with only one author or several with the same address
% ---------------
%\oneauthor
%   {\firstauthor} {Affiliation1 \\ %
%     {\tt \href{mailto:author1@smcnetwork.org}{author1@smcnetwork.org}}}

%Two addresses
%--------------
% \twoauthors
%   {\firstauthor} {Affiliation1 \\ %
%     {\tt \href{mailto:author1@smcnetwork.org}{author1@smcnetwork.org}}}
%   {\secondauthor} {Affiliation2 \\ %
%     {\tt \href{mailto:author2@smcnetwork.org}{author2@smcnetwork.org}}}

% Three addresses
% --------------
 \threeauthors
   {\firstauthor} {Zurich University of the Arts\\
   Institute for Computer Music\\ and Sound Technology\\
     {\tt \href{jan.schacher@zhdk.ch}{jan.schacher@zhdk.ch}}}
   {\secondauthor} {Affiliation2 \\ %
     {\tt \href{mailto:me@chikashi.net}{author2@smcnetwork.org}}}
   {\thirdauthor} { Affiliation3 \\ %
     {\tt \href{mailto:author3@smcnetwork.org}{author3@smcnetwork.org}}}


% ***************************************** the document starts here ***************
\begin{document}
%
\capstartfalse
\maketitle
\capstarttrue
%
\begin{abstract}
The development and specification of SpatDIF, the Spatial Sound Description Interchange Format, is complemented with actual implementations in software in order to become available in various audio software environments.
This article discusses the current state of the development of a software library called `libspatdif', whose purpose is to provide a reference implementation of SpatDIF and demonstrate its best-use practices.
In addition, the design principles derived from the concepts and specifications of SpatDIF, the class structure of the software library, and the concrete implementations demonstrating its usage in computer music applications is presented.
\end{abstract}
%
\section{Background}

SpatDIF, the Spatial Sound Description Interchange Format, presents a structured syntax for describing spatial audio information, addressing the different tasks involved in creating and performing spatial sound scenes.
The goal of the SpatDIF approach is to simplify and enhance the methods of creating spatial sound content and to enable the exchange of between otherwise incompatible software. 
SpatDIF proposes a simple and extensible format as well as best-practice exmaples for storing and transmitting spatial sound scene descriptions. 
It encourages portability and the exchange of compositions between venues with different surround audio infrastructures. 
SpatDIF also fosters collaboration between artists such as composers, musicians, sound installation artists, and sound designers, as well as researchers in the fields of acoustics, musicology, sound engineering and virtual reality.
SpatDIF is developed in a collaborative effort and has evolved over a number of years. 
The community and all related information can be found at \footnotesize{\url{www.spatdif.org}}.\normalsize

SpatDIF was coined in \citeyear{peters_caa07} \cite{peters_caa07} when Peters posited the need for a format describing spatial sound scenes in a structured way, since at that time all of the available spatial rendering systems used self-contained, proprietary syntax- and data-formats.
Through a panel discussion \cite{2008ICMCpanel, Peters:2008spatdif} and other meetings and workshops, the scope and concepts of SpatDIF were extended, refined, and consolidated.

After an extended process the SpatDIF specification was informally presented to the spatial sound community at the ICMC in Huddersfield in August 2011 and at a workshop at the TU-Berlin in September 2011.
The responses in these meetings suggested the urgent need for a lightweight and easy to implement spatial sound scene standard, which would contrast the complex MPEG-4 scene description specification \cite{scheirer1999audiobifs}.

The completion of a first usable version of the specifications \cite{SpatDIF_03} defining the core descriptors and a few indispensable additional descriptors was achieved the following year. The publication of this project milestone at the SMC in Copenhagen \cite{SpatDIF_SMC12} won a best paper award and was subsequently published in the Computer Music Journal \cite{Peters:2013SpatDifCMJ}.
This encouragement incited the group to tackle the task of proving that the concept would work in actual software.

A central principle for SpatDIF is the conceptual separation of authoring and rendering of spatial sound scenes or pieces. They may occur at separate times using the same or differing infrastructure, at separate places either simultaneously or with a long time between the two and of course through a mixture of these factors. 
The exact modality should not have to be determined at the outset.

In addition to the separation of the basic functionalities two principal use-cases can be distinguished.
The first scenario is focused on storing spatial audio pieces on a support for future playback. 
The second scenario deals with streamed content and scene information in real- or near real-time.
For these applications SpatDIF formulates a concise semantic structure that is capable of carrying all the information relevant for preserving a sound scene, without being tied to a specific implementation or technical method.
Since SpatDIF is a syntax rather than a programming interface or file-format it may be represented in any of the current or future structured mark-up languages or messaging systems.
It describes only those aspects required for the storage and transmission of \emph{spatial sound information}.
Because a complete work typically contains aspects that are outside the realm of spatial sound scenes, SpatDIF provides descriptors to link these aspects to the spatial dimensions, but only to the extent necessary.
For example, the description of media-streams in SpatDIF is necessary in order to define the storage location or origin of the audio present in the scene.
For example, to be able to render the audio objects with the correct audio content in the spatial scene, the storage location of the audio content needs to be defined via SpatDIF's media resources.
For example, the description of media-streams in the Media extension is necessary in order to describe from where the sounding content of the scene originates.

\section{libspatdif Concepts}


In this article we present the development and implementation of a software tool is aimed at an easy integration of SpatDIF into existing software.
The concepts and guidelines laid down in the SpatDIF specifications are implemented in a software-library.  
The use of this library is demonstrated in externals for MaxMSP and PureData and in an application written entirely in C++ in the creative coding environment openFrameworks \ref{of}.

After establishing a coherent specification with example use cases in textual form only, the next development step is the implementation of software that embodies the specified concepts.
For this purpose a platform-independent software library was designed and is now implemented. 


Additionally, to demonstrate the usage of SpatDIF, an example application which utilises this library was created. This application features SpatDIF file-handling and  rendering of SpatDIF sound scenes into multichannel sound files.

By providing a software library rather than just a complete software application, implementations in many different software environments are facilitated, which is one of the strategic goals of the project.


\section{Class Structure}\label{sec:class_structure}
Half page??

\section{Introduction to externals}
Even though the SpatDIF syntax is an implementation-independent specification, rather than a concrete software interface, the actual value of using it only becomes evident in concrete applications. Although SpatDIF was developed with a number of different usage scenarios in mind, the ones most closely associated with these authors' practices are electro-acoustic surround audio compositions for concerts or installations or real-time spatialisation in computer music performance.
Therefore the primary application for the SpatDIF library lies in an implementation in real-time capable audio software, such as MaxMSP or Pure Data.

\subsection{Concepts and Practical usage}\label{subsec:concepts}
In order to explore the methods and actual handling of the libspatdif in a real situation a dual testbed was implemented as externals for MaxMSP and Pure Data. In addition an example application with a limited feature-set was tested in OpenFrameworks, mainly to establish and test a workflow purely in C++.





\subsection{Playback}\label{sec:Playback}
example text

\subsection{Recording}\label{sub:body}
example text

\section{Future work}
example text


\begin{acknowledgments}
example text
\end{acknowledgments} 

%%%%%%%%%%%%%%%%%%%%%%%%%%%%%%%%%%%%%%%%%%%%%%%%%%%%%%%%%%%%%%%%%%%%%%%%%%%%%
%bibliography here
% \bibliography{SpatDIF.bib}
\printbibliography

\end{document}
