% JSSA Report Template for English ver.200908
% By Daichi Ando
% based on ICMC2005

\documentclass[a4paper]{article}
\renewcommand{\baselinestretch}{0.9}
\usepackage{jssa_e,amsmath}
% \usepackage{mediabb}
\usepackage{graphicx}
\usepackage{color}
\usepackage[]{biblatex}
\addbibresource{SpatDIF.bib}
\usepackage{listings}
\usepackage{enumitem}
\usepackage{balance}
% Title.
% ------

\title{spatdif library -- Implementing the Spatial Sound Descriptor Interchange Format}


% Paper Category
\category{Research Report}

% Single address
% To use with only one author or several with the same address
% ---------------
%\oneauthor
%  {Author Name} {Faculty of Fine
%  Arts, Tokyo University of the Arts}

% Two addresses
% --------------
\twoauthors
  {Chikashi Miyama} {College of Music Cologne\\Studio for Electronic Music\\ me@chikashi.net }
  {Jan C. Schacher} {Zurich University of the Arts\\ ICST \\ jan.schacher@zhdk.ch}

% Three addresses
% --------------
%\threeauthors
%  {First author} {School \\ Department}
%  {Second author} {Company \\ Address}
%  {Third author} {Company \\ Address}

\begin{document}
%
%%% -- Page Number Designation
% Ignore when submitting

\newcommand{\red}[1]{\textcolor{red}{#1}}
\newcommand{\todo}[1]{\noindent\textcolor{red}{[\underline{TODO}: #1]}}

\makeatletter 
\def\ps@myheadings{% 
\let\ps@jpl@in\ps@plain% 
\def\@evenhead{\reset@font\hfil\leftmark\hfil}% 
\def\@oddhead{\reset@font\hfil\rightmark\hfil}% 
\let\@mkboth\@gobbletwo% 
\let\sectionmark\@gobble% 
\let\subsectionmark\@gobble% 
% 
\def\@oddfoot{\reset@font\hfil-- \thepage --\hfil}% 
\let\@evenfoot\@oddfoot 
} 
\makeatother 
%%% 
%%% Designation of starting page number
% Ignore when submitting
\setcounter{page}{1} 
\pagestyle{myheadings} 
%%%
% Designation of header
% Ignore when submitting
\markright{\footnotesize \sl Journal of the Japanese Society for Sonic Arts, Vol.1 No.1 pp.17--21} 
%%% 
%%% \maketitle?????????????????? \thispagestyle{myheadings} ?????????????????? 
\maketitle
\thispagestyle{myheadings}
\sloppy
\setlist[itemize]{noitemsep, topsep=0pt}


\begin{abstract}

Here we have an abstract

\end{abstract}

% keywords ?

\section{Introduction}

In this article we present the development of a software tool aimed at simple integration of SpatDIF into existing software.
The concepts and guidelines are implemented in a C-Library and applied in an example surround-playback application.

SpatDIF, the Spatial Sound Description Interchange Format, presents a structured syntax for describing spatial sound information, addressing the different tasks involved in creating and performing spatial sound.

The goal of the SpatDIF approach is to simplify and enhance the methods of working with and exchanging spatial sound content. 
SpatDIF proposes a simple, minimal, and extensible format as well as best-practice implementations for storing and transmitting spatial sound scene descriptions. 
It encourages portability and the exchange of compositions between venues with different surround sound infrastructures. 
SpatDIF also fosters collaboration between artists such as composers, musicians, sound installation artists, and sound designers, as well as researchers in the fields of acoustics, musicology, sound engineering and virtual reality.

SpatDIF is developed as a collaborative effort and has evolved over a number of years. 
The community and all related information can be found at \url{www.spatdif.org}.

\section{History of the Project} %assigned to Jasch
SpatDIF was coined in \citeyear{peters_caa07} \cite{peters_caa07} when Peters stated the necessity for a format to describe spatial sound scenes in a structured way, since at that time the available spatial rendering systems all used self-contained syntax and data-formats. 
Through a panel discussion \cite{2008ICMCpanel, Peters:2008spatdif} and other meetings and workshops, the concept of SpatDIF has since been extended, refined, and consolidated. 

After a long and thoughtful process, the SpatDIF specification was informally presented to the spatial sound community at the ICMC in Huddersfield in August 2011, at a workshop at the TU-Berlin in September 2011.
The responses in these meetings suggested the urgent need for a lightweight and easy to implement spatial sound scene standard, which could contrast the complex MPEG specification \cite{scheirer1999audiobifs}.  
In addition, many features necessary to make this lightweight standard become functional were put forward in the 2013 Computer Music Journal article \cite{Peters:2013SpatDifCMJ}, such as for example the capability of dealing with temporal interpolation of scene descriptors as described.

\section{Concepts} %assigned to Jasch/Trond/Nils

One of the guiding principles for SpatDIF is the idea that authoring and rendering of spatial sound might occur at completely separate times and places, and be executed with tools whose capabilities cannot be known in advance. 
It formulates a concise semantic structure that is capable of carrying all the relevant information, without being tied to a specific implementation, thought-model or technical method. 
SpatDIF is a syntax rather than a programming interface or file-format and may be represented in any of the structured mark-up languages or message systems that are in use today or in the future. 
It describes only those aspects required for the storage and transmission of \emph{spatial information} regarding the sound.
However, a complete work typically contains additional dimensions that lie outside the scope of SpatDIF. 
These are judiciously addressed only to the extent necessary for linking these external sound elements to the descriptions of the spatial dimension. For example, the description of media-streams in the Media extension is necessary in order to describe fro where the sounding content of the scene originates.

After establishing a coherent specification with example use-cases in textual form only, the next step towards usability in a technological manner is the implementation of software that embodies the concepts and stores and executes the SpatDIF descriptors in actual surround sound.
For this purpose a software library was designed and is being implemented in a platform independent manner. 
In addition an example application that utilises this library is developed, in order to demonstrate the usage in a full form that generates actual sound output.
By providing a software library rather than just a complete software application, implementations in many different software environments are facilitated, which is one of the  strategic goals of the project.
 
Referring back to the canonical example ``Turenas'' by John Chowning (see also \cite{Peters:2013SpatDifCMJ}), the beginning of a SpatDIF example scene, including only the `insect' trajectory at second 0:44, contains the following elements in an XML format:

\lstinputlisting[tabsize=3,columns=fullflexible,breaklines=true,numbers=none,basicstyle=\scriptsize\ttfamily]{turenas-insect-extract.xml} 

Apart from the SpatDIF-compliant xml-file, or other structured markup formatted file, the corresponding sound-files have to be stored and transported alongside, in order to permit recreation of the scene.
It is therefore important to think in terms of `SpatDIF-bundles' or projects rther than single files.
A deliberate choice was made not to propose a format that combines soundfiles and scene descriptors in a binary format, since the readability without additional software tools would be lost.

\section{The Library} %assigned to Chikashi

In the following sections, a detailed description of the concepts and structures implemented in both the library and the example application are provided.
Although this information is mainly relevant to software developers implementing SpatDIF in software, we believe that showing these technical details also provides an additional perspective on the possibilities SpatDIF offers as a syntax.

\subsection{Features}
SpatDIFLib is an open source C/C++ multi-platform library, that offers the following functionalities to the developers of SpatDIF compatible softwares (clients).

\begin{itemize}
\item Loading and storing SpatDIF scenes from/to a XML, JSON, and YAML formatted string
\item Addition, deletion, and modification of entities in SpatDIF scenes
\item Addition, deletion, and modification of events
\item Associating events with entities
\item Activation and deactivation of extensions
\item Answering queries in regard to entities and events in SpatDIF scenes
\item Controlling data in SpatDIF Scenes via OSC Messages
\end{itemize}

Though SpatDIFLib can be controlled by OSC formatted strings, the library does not handle network sockets directly for the OSC communication; the library interprets given OSC strings as commands. The client application should prepare sockets and threads for OSC communication. Likewise no scheduler or timer is implemented in the library. 
  
\subsection{C++ Class Structure}

Figure \ref{fig:class_structure} shows simplified class hierarchy of the library. 

% explains what I want to show here 2 - 3sentences
A instances of {\it sdScene} class represents a SpatDIF scene and maintains instances of {\it sdEntityCore}. The functionalities of {\it sdEntityCore} may be extended by the descendants of {\it sdEntityExtension}. The activation and deactivation of the extension is global within a scene. Thus, sdScene is also responsible for the extension handling. Each instance of
sdEntityCore maintains instances of {\it sdEvent} that are ordered in time and represent events of entities that they are attached to. The followings are the brief descriptions of these most important classes.

\begin{figure*}[t]
\centerline{
	\includegraphics[width= 17.5cm]{classes.pdf}}
\caption{Class Structure}
\label{fig:class_structure}
\end{figure*}

\subsubsection{sdScene}
The class sdScene maintains all data associated to a SpatDIF scene. This class offers clients the following three functionalities.

\begin{itemize}
\item Addition, deletion and modification entities in its scene
\item Addition and modification of the meta data associated to its scene
\item Activation and deactivation of the extensions
\end{itemize}

Once the client activates an extension in a scene, sdScene automatically adds extended functionalities and allocates extra buffers to all existing and newly created instances of sdEntityCore. By the deactivation of an extension, sdScene removes all extended functionalities and previously allocated buffers of all existing sdEntities. Subsequently, all data stored in the extension buffers are discarded.

\subsubsection{sdLoader/sdSaver}
These two classes provide several utility functions and enable clients to create an instance of sdScene from a XML, JSON, or YAML string and vice versa. In order to maintain platform independency and to achieve maximum flexibility, the library does not handle files directly; The client is responsible for the file management. 

These functions utilise two external libraries for parsing markup formatted strings. TinyXML-2\footnote{http://www.grinninglizard.com/tinyxml/ accessed Oct. 9. 2013} and libjson\footnote{http://sourceforge.net/projects/libjson/ accessed Oct. 9. 2013}.

\subsubsection{sdEvent}
This is a pure abstract class of event, that maintains following three data items.

\begin{itemize}
\item time - absolute time of the event
\item descriptor - type of event
\item value - actual data
\end{itemize}

\subsubsection{sdEntity}
This is a pure abstract class of entity in SpatDIF scene. All basic functionalities, such as addition, deletion, and modification of events are implemented or declared as pure virtual functions.

\subsubsection{sdEntityCore}
An instance of sdEntityCore maintains events with SpatDIF core descriptors and has a vector to hold instances of SpatDIF extensions. This class is also responsible for answering queries from the client conceding its events. For example, if a client asks an instance of sdEntityCore a value of a certain descriptor at a specific time, the sdEntity returns value to the client. The client is able to ask for multiple events within a certain time frame and filter events by descriptor. If the client requests values of extended descriptors, the instance of sdEntityCore forwards the query to the attached extension.

\subsubsection{sdEntityExtension}
This is a pure abstract class of extensions. The descendants of this class. e.g. sdEntityExtensionMedia handles the events with extended descriptors. If the client activates an extension in a scene, all existing instances of sdEntityCore create an instance of the designated subclass of sdEntityExtension and register it to their internal vector.

\subsection{Simple Example}
The following code listing shows how to load an XML-formatted string into an sdScene and query the entity called ``insect'' for the first occurrence of an event which contains a position- and a media-descriptor.

\vfill

\lstinputlisting[columns=fullflexible,breaklines=true,numbers=left,basicstyle=\scriptsize\ttfamily]{example_code2.cpp} 

\noindent This produces the following std-output: 
\lstinputlisting[columns=fullflexible,breaklines=true,numbers=none,basicstyle=\scriptsize\ttfamily]{code_output.txt} 

L1: sdLoader::sceneFrom static function loads a SpatDIF scene from a XML formatted string. 
L2: A pointer to an entity, named "insect", is obtained by scene.getEntity function.
L3-4: The entity "insect" is requested to return a pointer to the first event with the position and media location descriptor.
L6: Querying the "insect" entity for its name
L7-9: Posting values and time of events 

\subsection{Future Work}

The above mentioned basic class hierarchy is already implemented in the library. The library is also able to interpret simple XML, JSON and OSC messages and is currently examined against  the SpatDIFRenderer, in order to further improve its performance.  All extensions defined in the SpatDIF specification 0.3 will be implemented in early 2014 and a C interface will be added in late 2014. The library will be completed by the end of 2014 and released under the MIT, BSD or similar licenses.

\section{The Example Application}% Jasch

An example application is developed, that implements an entire workflow for playback of SpatDIF files. 
It is called a `renderer' in analogy to visual tools, because it renders audible, in a surround setup, the information contained in a SpatDIF `bundle'.
It also serves to validate the development of the library, in the sense of a complex test-case that reflects real-life usage.
The implementation has to solve all the question relating to file-handling, OSC-streams, instantiating playback, panning, distance cues and the other descriptors present in the specifications 0.3.

More importantly however, it demonstrates the power and simplicity of SpatDIF, by showing how one of the most common use-cases is tackled.

\subsection{Scope}

In order to provide a relevant example for the application of the libspatdif-library, the scope of the application has deliberately been limited.
The panning algorithm is a simple spatial windowing algorithms called ``ambipanning'' \cite{Neukom:2008ambipan} that has the advantage of being a one-step calculation (contrary to ambisonics), which is highly flexible, easy to implement, not tied to a specific number of speakers and usable without modification both in two and three dimensional spatialisation situations.

\subsection{Implementation details}

This application is implemented in the creative-coding environment openFrameworks\footnote{http://www.openframeworks.cc/ URI valid 2013/10/23}, which provides a powerful C++ toolset and a thriving community.
Since it is not particularly oriented towards sound programming, the provided classes are somewhat rudimentary.
However -- and that is its great strength -- many extensions exist and it is releasy to add new functionalities and libraries.
One one of these extension-libraries that is being used for this application is libsndfile  
\footnote{http://www.mega-nerd.com/libsndfile/ URI valid 2013/10/23} a powerful audiofile handling toolset.
 
Figure \ref{fig:screenshot} shows the scene-display of the example application. 

\begin{figure}[h]
\centerline{
	\includegraphics[width=\columnwidth]{SpatDIFrenderer_screenshot2}}
\caption{Interface of the example renderer application.\todo{export from OF as SVG/PDF}}
\label{fig:screenshot}
\end{figure}

 



\subsection{Distinction to Library}



ref openFrameworks
ref libsndfile


(rough sketch....)

app scope / what does it do

app interfaces to lib $\rightarrow$ task distribution between app and lib

i.e. scheduler

or sockets / file-IO
\section{Availability}% Jasch

Both he library and the example application will be made publicly and freely available, as soon as the full feature-set of the 0.3 specifications are implemented a thoroughly tested.
But of course, an advance access to the code can be granted on demand.

\section{Acknowledgements}% Jasch

This software development would not have been possible without the generous financial support of the Institute for Computer Music and Sound Technology ICST of the Zurich University of the Arts.


\printbibliography


\section{Author's Profile}
\balance

\subsection*{Chikashi Miyama}
Chikashi Miyama is a composer, video artist, interface designer, performer, and author. He received a MA (Sonology/2004) from Kunitachi College of Music, Tokyo, Japan, a Nachdiplom (Komposition im Elektronischen Studio/2007) from Music academy of Basel, Switzerland, and a Ph.D (Composition/2011) from University at Buffalo, New york, USA. His compositions have received an ICMA award (2011/UK) from the International Computer Music Association, a second prize in SEAMUS commission competition (2010/St. Cloud, USA), a special prize in Destellos Competition (2009/Argentina), and a honorable mention in the Bourges Electroacoustic Music Competition (2002/France). Several works of him are included on the DVD of the Computer Music Journal Vol.28 by MIT press, and ICMC official CD/DVD(2005/2011). In 2011, he received a research grant from DAAD (German Academic Exchange Service) and worked as a visiting researcher at ZKM, Karlsruhe, Germany. He has taught computer music at  University at Buffalo, USA,  Music Academy Basel, and College of Arts Bern, Switzerland. He is currently teaching at College of Music and Dance, Cologne, Germany. 

\subsection*{Jan C. Schacher}
A doublebass-player, composer and digital artist, Jan Schacher is active in electronic and exploratory music, in jazz, contemporary music, performance and installation art as well as writing music for chamber-ensembles, theatre and film. His main focus is on works combining digital sound and images, abstract graphics and experimental video in the field of electro-acoustic music and in mixed-media projects for the stage and in installations. Jan Schacher has been invited as artist and lecturer to numerous cultural and academic institutions and has presented installations in galleries and performances in clubs and at festivals such as R?sonance Festival (Paris), Sonar Festival (Barcelona), Transmediale Festival (Berlin), the Holland Festival (Amsterdam) and many other venues throughout Europe, North America, Australia and Asia.


\end{document}



% TODO spatfdif-lib paper
% 
% - history  of the project Jasch
% - conceptual explanation / structure Jasch/Trond/ 
% - library details/structure Chikashi
% - reference renderer (rough sketch....) Jasch
% 
% JSSA presentation dec 7. 
% paper deadline nov 7.
% 
% paper finished nov 4.
% 
% milestones:
% 7. october first sketches structure, (blocks of text)
% 21. october finished textblocks
% 4. november finished with all the refs. corrected submittable
% 




Use basic \LaTeX commands.

Headings are as follows:

\subsection{subsection}

This is \verb|\subsection{}|.

\subsubsection{subsubsection}

This is \verb|\subsubsection{}|.

\section{Include Figures}

The way to insert figures is as follows:

\begin{figure}[h]
\centerline{
	\includegraphics[mediaboxonly,width=\columnwidth]{figure.pdf}}
\caption{English Caption.}
\label{fig:figure}
\end{figure}

Giving [h] option, insert a figure at the specified position.

\section{Foot Note}

The way to insert footnote is as follows\footnote{This is foot note}.

\section{Citiation}

This is citiation\cite{Author1:08}.

Also multiple citiation is as follows\cite{Author1:08,Author2:09}.

\section{For Count Number of Characters}


\bibliographystyle{plain}
\begin{thebibliography}{citations}
\bibitem{Author1:08} Henoheno Moheji {\it The Book of Computer Music},
	Mingmei Publishing, 2008.
\bibitem{Author2:09} Harahoro???Hirehare ``The Abstract of The Book of
	Computer Music'', {\it Comtemporary Computer Music Society},
	Vol.2 No.4, pp22-26, 2009.
\end{thebibliography}


\end{document}

=======
>>>>>>> External Changes
