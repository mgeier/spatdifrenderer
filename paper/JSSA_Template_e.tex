% JSSA Report Template for English ver.200908
% By Daichi Ando
% based on ICMC2005

\documentclass{article}
\renewcommand{\baselinestretch}{0.9}
\usepackage{jssa_e,amsmath}
\usepackage{mediabb}
\usepackage{graphicx}

\usepackage[]{biblatex}
\addbibresource{SpatDIF.bib}

% Title.
% ------
\title{Implementing the Spatial Sound Descriptor Interchange Format SpatDIF}

% Paper Category
\category{Short Paper}

% Single address
% To use with only one author or several with the same address
% ---------------
%\oneauthor
%  {Author Name} {Faculty of Fine
%  Arts, Tokyo University of the Arts}

% Two addresses
% --------------
\twoauthors
  {Chikashi Miyama} {College of Music Cologne\\Studio for Electronic Music \\ Unter Krahnenb\"{a}umen 87, Cologne, Germany\\ me@chikashi.net }
  {Jan C. Schacher} {Zurich University of the Arts\\Institute for Computer Music and Sound Technology\\Baslerstrasse 30, Zurich, Switzerland\\jan.schacher@zhdk.ch}

% Three addresses
% --------------
%\threeauthors
%  {First author} {School \\ Department}
%  {Second author} {Company \\ Address}
%  {Third author} {Company \\ Address}

\begin{document}
%
%%% -- Page Number Designation
% Ignore when submitting

\makeatletter 
\def\ps@myheadings{% 
\let\ps@jpl@in\ps@plain% 
\def\@evenhead{\reset@font\hfil\leftmark\hfil}% 
\def\@oddhead{\reset@font\hfil\rightmark\hfil}% 
\let\@mkboth\@gobbletwo% 
\let\sectionmark\@gobble% 
\let\subsectionmark\@gobble% 
% 
\def\@oddfoot{\reset@font\hfil-- \thepage --\hfil}% 
\let\@evenfoot\@oddfoot 
} 
\makeatother 
%%% 
%%% Designation of starting page number
% Ignore when submitting
\setcounter{page}{17} 
\pagestyle{myheadings} 
%%%
% Designation of header
% Ignore when submitting
\markright{\footnotesize \sl Journal of the Japanese Society for Sonic Arts, Vol.1 No.1 pp.17--21} 
%%% 
%%% \maketitle?????????????????? \thispagestyle{myheadings} ?????????????????? 
\maketitle
\thispagestyle{myheadings}

\begin{abstract}

Here we have an abstract

\end{abstract}

% keywords ?

\section{Introduction}

In this article we present the development of a tool for easy integration of SpatDIF into existing software.
The concepts and guidelines are implemented in a C-Library and applied in an example surround-playback application.

SpatDIF, the Spatial Sound Description Interchange Format, presents a structured approach for working with spatial sound information, addressing the different tasks involved in creating and performing spatial sound.

The goal of SpatDIF is to simplify and enhance the methods of working with spatial sound content. 
SpatDIF proposes a simple, minimal, and extensible format as well as best-practice implementations for storing and transmitting spatial sound scene descriptions. 
It encourages portability and the exchange of compositions between venues with different surround sound infrastructures. 
SpatDIF also fosters collaboration between artists such as composers, musicians, sound installation artists, and sound designers, as well as researchers in the fields of acoustics, musicology, sound engineering and virtual reality.

SpatDIF is developed as a collaborative effort and has evolved over a number of years. 
The community and all related information can be found at \url{www.spatdif.org}.

\section{history of the project} %assigned to Jasch
SpatDIF was coined in \cite{peters_caa07} when Peters stated the necessity for a format to describe spatial sound scenes in a structured way, since at that time the available spatial rendering systems all used self-contained syntax and data-formats. 
Through a panel discussion \cite{2008ICMCpanel, Peters:2008spatdif} and other meetings and workshops, the concept of SpatDIF has since been extended, refined, and consolidated. 

After a long and thoughtful process, the SpatDIF specification was informally presented to the spatial sound community at the ICMC 2011 in Huddersfield in August 2011, at a workshop at the TU-Berlin in September 2011 and in its current form in a Computer Music Journal article in 2013 \cite{}
The responses in these meetings suggested the urgent need for a lightweight and easy to implement spatial sound scene standard, which could contrast the complex MPEG specification \cite{scheirer1999audiobifs}.  
In addition, many features necessary to make this lightweight standard functional were put forward, for example the capability of dealing with temporal interpolation of scene descriptors. 

\section{conceptual explanation / structure} %assigned to Jasch/Trond/Nils

One of the guiding principles for SpatDIF is the idea that authoring and rendering of spatial sound might occur at completely separate times and places, and be executed with tools whose capabilities cannot be known in advance. 
It formulates a concise semantic structure that is capable of carrying the necessary information, without being tied to a specific implementation, thought-model or technical method. 
SpatDIF is a syntax rather than a programming interface or file-format and may be represented in any of the structured mark-up languages or message systems that are in use today or in the future. 
It describes only the aspects required for the storage and transmission of \emph{spatial information}.
A complete work typically contains additional dimensions that lie outside the scope of SpatDIF. 
These are only addressed to the extent necessary for linking the elements to the descriptions of the spatial dimension (i.e. the Media extension).

explaining the library idea: WHY a library

score-example in XML / JSON -> turenas xml score

format of SpatDIF `bundle'


\section{library details/structure} %assigned to Chikashi

\subsection{Features}
SpatdifLib is an open source C/C++ multi-platform library, that offers the following functionalities to the developers of SpatDIF compatible softwares (clients).
\begin{itemize}
\item loading and storing SpatDIF scenes from/to a XML, JSON formatted string.
\item addition, deletion, and modification of entities in SpatDIF scenes
\item addition, deletion, and modification of events and associate them to the existing entities in SpatDIF scenes
\item activation and deactiovation of extensions
\item answering queries about the data stored in a scene
\item realtime data handling from a remote host via OSC protocol
\end{itemize}

In order to maintain platform independency and provide clients with maximum flexibility, the library does not handle XML or JSON files directly. Instead, it interprets provideded XML, JSON or YAML formatted string, employing Mit Licensed parsers, TinyXML2 \footnote{http://www.grinninglizard.com/tinyxml/ accessed Oct. 9. 2013} and libjson \footnote{http://sourceforge.net/projects/libjson/ accessed Oct. 9. 2013}.

  Schedulers or timers are not implemented in SpatDIF library; all these functionalities should be implemented in the client application. 
  
  Though SpatDIFLib is controllable employing OSC formatted string, it does not handle network sockets directly for OSC communication; the library simply interprets provided OSC string as commands. The client application should prepare sockets for OSC communication.

\subsection{C++ Class Structure}
Figure XXX shows simplified class hierarchy of the library. 

\subsubsection{sdScene}
sdScene maintains all data associated to a certain SpatDIF scene. This class offers clients the following three functionalities

\begin{enumerate}
\item adding, removing and modifying entities in its scene
\item activation and deactivation of the extensions
\item editing meta data associated to the scene
\end{enumerate}

Once the client activates an extension in a scene, sdScene automatically add extended functionality and storage to all existing and newly created instances of sdEntityCore. By the deactivation of an extension, sdScene removes all extended functionality and previously allocated storage of all existing sdEntities. Subsequently, all the data associated to the extension will be discarded. 

\subsubsection{sdLoader/sdSaver}
These two classes provide several utility functions and enable clients to easily convert a XML, JSON, or YAML string to a sdScene and vice versa,

\subsubsection{sdEntityCore}
An instance of sdEntityCore maintains events with SpatDIF core descriptors and keep a vector to hold instances of SpatDIF extensions. This class is also responsible for answering query from the client about events. For example, if a client asks an entity a value of a certain descriptor at a specific time, the sdEntity returns value to the client. The client is able to request multiple events within a certain time frame and filter events by descriptor. If the client request values of extended desciriptors, the sdEntityCore forwards the query to the attached extension.

\subsubsection{sdEntityExtension}
This is a pure abstract class of extensions. The subclasses of this class. e.g. sdEntityExtensionMedia handle the events with extended descriptors. If the cleint activates an extension in a scene, all existing sdEntityCore create an instance of designated subclass of sdEntityExtension and register it to the internal vector.

\subsection{Simple Example}
The following example code shows how to load a XML formatted string to a sdScene and query a event stored in it.

How to query.cpp
$\rightarrow$ turenas xml score C++ example

\subsection{Future plan}

The above mentioned basic class hierarchy is already implemented in the library. The library is also able to interpret simple XML, JSON and OSC messages and currently examined against SpatDIFRenderer, in order to further improve flexibility and practicality.  All extensions defined in the SpatDIF specification 0.3 will be implemented in early 2014 and C interface will be added in the late 2014. The library will be completed by the end of 2014 and released under the MIT or similar license.

\section{reference renderer }% Jasch

ref openFrameworks
ref libsndfile


(rough sketch....)

app scope / what does it do

app interfaces to lib $\rightarrow$ task distribution between app and lib

i.e. scheduler

or sockets / file-IO

\printbibliography


\section{Author's Profile}
\subsection*{Chikashi Miyama}
Chikashi Miyama is a composer, video artist, interface designer, performer, and author. He received a MA (Sonology/2004) from Kunitachi College of Music, Tokyo, Japan, a Nachdiplom (Komposition im Elektronischen Studio/2007) from Music academy of Basel, Switzerland, and a Ph.D (Composition/2011) from University at Buffalo, New york, USA. His compositions have received an ICMA award (2011/UK) from the International Computer Music Association, a second prize in SEAMUS commission competition (2010/St. Cloud, USA), a special prize in Destellos Competition (2009/Argentina), and a honorable mention in the Bourges Electroacoustic Music Competition (2002/France). Several works of him are included on the DVD of the Computer Music Journal Vol.28 by MIT press, and ICMC official CD/DVD(2005/2011). In 2011, he received a research grant from DAAD (German Academic Exchange Service) and worked as a visiting researcher at ZKM, Karlsruhe, Germany. He has taught computer music at  University at Buffalo, USA,  Music Academy Basel, and College of Arts Bern, Switzerland. He is currently teaching at College of Music and Dance, Cologne, Germany. 

\subsection*{Jan C. Schacher}
A doublebass-player, composer and digital artist, Jan Schacher is active in electronic and exploratory music, in jazz, contemporary music, performance and installation art as well as writing music for chamber-ensembles, theatre and film. His main focus is on works combining digital sound and images, abstract graphics and experimental video in the field of electro-acoustic music and in mixed-media projects for the stage and in installations. Jan Schacher has been invited as artist and lecturer to numerous cultural and academic institutions and has presented installations in galleries and performances in clubs and at festivals such as R?sonance Festival (Paris), Sonar Festival (Barcelona), Transmediale Festival (Berlin), the Holland Festival (Amsterdam) and many other venues throughout Europe, North America, Australia and Asia.


\end{document}



% TODO spatfdif-lib paper
% 
% - history  of the project Jasch
% - conceptual explanation / structure Jasch/Trond/Nils
% - library details/structure Chikashi
% - reference renderer (rough sketch....) Jasch
% 
% JSSA presentation dec 7. 
% paper deadline nov 7.
% 
% paper finished nov 4.
% 
% milestones:
% 7. october first sketches structure, (blocks of text)
% 21. october finished textblocks
% 4. november finished with all the refs. corrected submittable
% 




Use basic \LaTeX commands.

Headings are as follows:

\subsection{subsection}

This is \verb|\subsection{}|.

\subsubsection{subsubsection}

This is \verb|\subsubsection{}|.

\section{Include Figures}

The way to insert figures is as follows:

\begin{figure}[h]
\centerline{
	\includegraphics[mediaboxonly,width=\columnwidth]{figure.pdf}}
\caption{English Caption.}
\label{fig:figure}
\end{figure}

Giving [h] option, insert a figure at the specified position.

\section{Foot Note}

The way to insert footnote is as follows\footnote{This is foot note}.

\section{Citiation}

This is citiation\cite{Author1:08}.

Also multiple citiation is as follows\cite{Author1:08,Author2:09}.

\section{For Count Number of Characters}


\bibliographystyle{plain}
\begin{thebibliography}{citations}
\bibitem{Author1:08} Henoheno Moheji {\it The Book of Computer Music},
	Mingmei Publishing, 2008.
\bibitem{Author2:09} Harahoro???Hirehare ``The Abstract of The Book of
	Computer Music'', {\it Comtemporary Computer Music Society},
	Vol.2 No.4, pp22-26, 2009.
\end{thebibliography}


\end{document}

